\documentclass[11pt, oneside]{article}   	% use "amsart" instead of "article" for AMSLaTeX format
\usepackage[utf8]{inputenc}
\usepackage[T1]{fontenc}
\usepackage[margin=1.0in]{geometry}
\geometry{letterpaper}                   	% ... or a4paper or a5paper or ... 

\usepackage{wrapfig}                    % wrap text around figures
\usepackage{graphicx}				    % Use pdf, png, jpg, or eps with pdflatex
\usepackage{dcolumn}                    % Align table columns on decimal point
\usepackage{amssymb, amstext, amsmath}  % blackboard math symbols
\usepackage{bm}                         % bold math
\usepackage{nicefrac}                   % compact symbols for 1/2, etc.
\usepackage{tabularx,booktabs}          % professional-quality tables
\usepackage{soul,xcolor}                % coloring and strikethrough fonts
\usepackage{verbatim}                   % block comments in text
\usepackage{float,placeins}             % precision placement of figures
\usepackage{hyperref}                   % links and urls

\setlength{\parindent}{1cm}
\setlength{\parskip}{5pt}

\begin{document}

%%%%%%%%%% Lagevin Equation (Introdunction to SDE's)

\section{Brownian Motion and The Langevin Equation}

    Brownian motion is one the simplest and most fundamental of all stochastic processes. While it has been given a formal mathematical definition in the many years since its original conception, we will begin not with the formal definitions, but from the physical motivations. Brownian motion was originally conceived as the motion of a large particle suspended in a solution of much smaller particles, like a grain of pollen in a bath of water molecules. The original analysis of this motion was given by Einstein in his famous 1905 paper \cite{Einstein}. Einstein noted that the mechanics of these Brownian particles could be described by a diffusion equation, and noted a powerful relationship between the diffusion constant and the size of the particles:
    \begin{equation}
        \label{eq:stokes_einstein}
        D = \frac{k_B T}{6\pi \eta r}
    \end{equation}
    where $\eta$ is the viscosity of the liquid. This method gave a way of measuring the size of Brownian particles, by first measuring the diffusion coefficient and viscosity. Einstein even suggested it as a way of measuring atomic sizes. 
    
    A deeper mathematical picture was provided by Paul Langevin, published no more than 3 years after Einstein's original result \cite{Langevin}. Rather than describing the ensemble motions of the particles through a diffusion equation, which was already well-understood mathematics by the time of Einstein, Langevin sought a microscopic picture where the motion of a single Brownian particle could be described by an ordinary differential equation. 
    \begin{equation}
        \label{eq:langevin}
        m\frac{dv}{dt} = -\gamma v - \frac{dU}{dx} + f(t)
    \end{equation}
    Here $v$ is the velocity of the particle, $\gamma$ is the friction coefficient of the medium, $- \frac{dU}{dx}$ is the external force, and $f(t)$ is a random fluctuating force. Note that this equation is very similair to the standard version of Newton's second law in a friction-inducing medium, except that it has an additional random \textbf{stochastic} term as well. 
    
\end{document}